\chapter{Metodologia}
\label{chap:metodologia}

Lorem ipsum dolor sit amet, consectetur adipiscing elit. Pellentesque in consectetur quam. Duis a mi a mauris tempus pretium non sit amet dui. Ut malesuada ex at purus rhoncus fermentum. Pellentesque auctor diam ut mauris vestibulum, aliquet feugiat odio scelerisque. Lorem ipsum dolor sit amet, consectetur adipiscing elit. Donec feugiat enim sed elit facilisis molestie. Suspendisse sed nisl at ligula porttitor mattis.

O autor \cite{lamport1986latex} e \cite{Maia2011} Lorem ipsum dolor sit amet, consectetur adipiscing elit. Pellentesque in consectetur quam. Duis a mi a mauris tempus pretium non sit amet dui. Ut malesuada ex at purus rhoncus fermentum. Pellentesque auctor diam ut mauris vestibulum, aliquet feugiat odio scelerisque. Lorem ipsum dolor sit amet, consectetur adipiscing elit. Donec feugiat enim sed elit facilisis molestie. Suspendisse sed nisl at ligula porttitor mattis.

\cite{Huetal2000} lorem ipsum dolor sit amet, consectetur adipiscing elit. Pellentesque in consectetur quam. Duis a mi a mauris tempus pretium non sit amet dui. Ut malesuada ex at purus rhoncus fermentum. Pellentesque auctor diam ut mauris vestibulum, aliquet feugiat odio scelerisque. Lorem ipsum dolor sit amet, consectetur adipiscing elit. Donec feugiat enim sed elit facilisis molestie. Suspendisse sed nisl at ligula porttitor mattis.

\section{Exemplo de Algoritmos e Figuras}
\label{sec:exemplo-de-algoritmos-e-figuras}

Lorem ipsum dolor sit amet, consectetur adipiscing elit. Pellentesque in consectetur quam. Duis a mi a mauris tempus pretium non sit amet dui. Ut malesuada ex at purus rhoncus fermentum. Pellentesque auctor diam ut mauris vestibulum, aliquet feugiat odio scelerisque. Lorem ipsum dolor sit amet, consectetur adipiscing elit. Donec feugiat enim sed elit facilisis molestie. Suspendisse sed nisl at ligula porttitor mattis.

\begin{figure}[h!]
    \centering
    \Caption{\label{fig:codigo-exemplo} Exemplo de código JavaScript}
    \UFERSAfig{}{
        \fbox{\begin{minipage}{15cm}
        \lstinputlisting[language=JavaScript]{codigos/exemplo.js}
        \end{minipage}}
    }{
        \Fonte{O Autor, 2018}
    }
\end{figure}

Lorem ipsum dolor sit amet, consectetur adipiscing elit. Pellentesque in consectetur quam. Duis a mi a mauris tempus pretium non sit amet dui. Ut malesuada ex at purus rhoncus fermentum. Pellentesque auctor diam ut mauris vestibulum, aliquet feugiat odio scelerisque. Lorem ipsum dolor sit amet, consectetur adipiscing elit. Donec feugiat enim sed elit facilisis molestie. Suspendisse sed nisl at ligula porttitor mattis.

\section{Usando Fórmulas Matemáticas}

Lorem ipsum dolor sit amet, consectetur adipiscing elit. Pellentesque in consectetur quam. Duis a mi a mauris tempus pretium non sit amet dui. Ut malesuada ex at purus rhoncus fermentum. Pellentesque auctor diam ut mauris vestibulum, aliquet feugiat odio scelerisque. Lorem ipsum dolor sit amet, consectetur adipiscing elit. Donec feugiat enim sed elit facilisis molestie. Suspendisse sed nisl at ligula porttitor mattis.

	\begin{equation}
		\begin{aligned}
			x = a_0 + \cfrac{1}{a_1
				+ \cfrac{1}{a_2
					+ \cfrac{1}{a_3 + \cfrac{1}{a_4} } } }
		\end{aligned}
	\end{equation}

Lorem ipsum dolor sit amet, consectetur adipiscing elit. Pellentesque in consectetur quam. Duis a mi a mauris tempus pretium non sit amet dui. Ut malesuada ex at purus rhoncus fermentum. Pellentesque auctor diam ut mauris vestibulum, aliquet feugiat odio scelerisque. Lorem ipsum dolor sit amet, consectetur adipiscing elit. Donec feugiat enim sed elit facilisis molestie. Suspendisse sed nisl at ligula porttitor mattis.

	\begin{equation}
		\begin{aligned}
			k_{n+1} = n^2 + k_n^2 - k_{n-1}
		\end{aligned}
	\end{equation}

Lorem ipsum dolor sit amet, consectetur adipiscing elit. Pellentesque in consectetur quam. Duis a mi a mauris tempus pretium non sit amet dui. Ut malesuada ex at purus rhoncus fermentum. Pellentesque auctor diam ut mauris vestibulum, aliquet feugiat odio scelerisque. Lorem ipsum dolor sit amet, consectetur adipiscing elit. Donec feugiat enim sed elit facilisis molestie. Suspendisse sed nisl at ligula porttitor mattis.

	\begin{equation}
		\begin{aligned}
			\cos (2\theta) = \cos^2 \theta - \sin^2 \theta
		\end{aligned}
	\end{equation}

Lorem ipsum dolor sit amet, consectetur adipiscing elit. Pellentesque in consectetur quam. Duis a mi a mauris tempus pretium non sit amet dui. Ut malesuada ex at purus rhoncus fermentum. Pellentesque auctor diam ut mauris vestibulum, aliquet feugiat odio scelerisque. Lorem ipsum dolor sit amet, consectetur adipiscing elit. Donec feugiat enim sed elit facilisis molestie. Suspendisse sed nisl at ligula porttitor mattis.

	\begin{equation}
		\begin{aligned}
			A_{m,n} =
			\begin{pmatrix}
			a_{1,1} & a_{1,2} & \cdots & a_{1,n} \\
			a_{2,1} & a_{2,2} & \cdots & a_{2,n} \\
			\vdots  & \vdots  & \ddots & \vdots  \\
			a_{m,1} & a_{m,2} & \cdots & a_{m,n}
			\end{pmatrix}
		\end{aligned}
	\end{equation}

Lorem ipsum dolor sit amet, consectetur adipiscing elit. Pellentesque in consectetur quam. Duis a mi a mauris tempus pretium non sit amet dui. Ut malesuada ex at purus rhoncus fermentum. Pellentesque auctor diam ut mauris vestibulum, aliquet feugiat odio scelerisque. Lorem ipsum dolor sit amet, consectetur adipiscing elit. Donec feugiat enim sed elit facilisis molestie. Suspendisse sed nisl at ligula porttitor mattis.

	\begin{equation}
		\begin{aligned}
			f(n) = \left\{
			\begin{array}{l l}
			n/2 & \quad \text{if $n$ is even}\\
			-(n+1)/2 & \quad \text{if $n$ is odd}
			\end{array} \right.
		\end{aligned}
	\end{equation}

Lorem ipsum dolor sit amet, consectetur adipiscing elit. Pellentesque in consectetur quam. Duis a mi a mauris tempus pretium non sit amet dui. Ut malesuada ex at purus rhoncus fermentum. Pellentesque auctor diam ut mauris vestibulum, aliquet feugiat odio scelerisque. Lorem ipsum dolor sit amet, consectetur adipiscing elit. Donec feugiat enim sed elit facilisis molestie. Suspendisse sed nisl at ligula porttitor mattis.

\section{Usando Código-fonte}

Lorem ipsum dolor sit amet, consectetur adipiscing elit. Pellentesque in consectetur quam. Duis a mi a mauris tempus pretium non sit amet dui. Ut malesuada ex at purus rhoncus fermentum. Pellentesque auctor diam ut mauris vestibulum, aliquet feugiat odio scelerisque. Lorem ipsum dolor sit amet, consectetur adipiscing elit. Donec feugiat enim sed elit facilisis molestie. Suspendisse sed nisl at ligula porttitor mattis.

\begin{figure}[h!]
    \centering
    \Caption{\label{fig:codigo-exemplo} Exemplo de código Java}
    \UFERSAfig{}{
        \fbox{\begin{minipage}{15cm}
        \lstinputlisting[language=Java]{codigos/exemplo.java}
        \end{minipage}}
    }{
        \Fonte{O Autor, 2018}
    }
\end{figure}

Lorem ipsum dolor sit amet, consectetur adipiscing elit. Pellentesque in consectetur quam. Duis a mi a mauris tempus pretium non sit amet dui. Ut malesuada ex at purus rhoncus fermentum. Pellentesque auctor diam ut mauris vestibulum, aliquet feugiat odio scelerisque. Lorem ipsum dolor sit amet, consectetur adipiscing elit. Donec feugiat enim sed elit facilisis molestie. Suspendisse sed nisl at ligula porttitor mattis.

\section{Usando Teoremas, Proposições, etc}

Lorem ipsum dolor sit amet, consectetur adipiscing elit. Nunc dictum sed tortor nec viverra. consectetur adipiscing elit. Nunc dictum sed tortor nec viverra.

\begin{teo}[Pitágoras]
	Em todo triângulo retângulo o quadrado do comprimento da
	hipotenusa é igual a soma dos quadrados dos comprimentos dos catetos.
\end{teo}


Lorem ipsum dolor sit amet, consectetur adipiscing elit. Nunc dictum sed tortor nec viverra. consectetur adipiscing elit. Nunc dictum sed tortor nec viverra.

\begin{teo}[Fermat]
	Não existem inteiros $n > 2$, e $x, y, z$ tais que $x^n + y^n = z$
\end{teo}

Lorem ipsum dolor sit amet, consectetur adipiscing elit. Nunc dictum sed tortor nec viverra. consectetur adipiscing elit. Nunc dictum sed tortor nec viverra.

\begin{prop}
	Para demonstrar o Teorema de Pitágoras...
\end{prop}

Lorem ipsum dolor sit amet, consectetur adipiscing elit. Nunc dictum sed tortor nec viverra. consectetur adipiscing elit. Nunc dictum sed tortor nec viverra.

\begin{exem}
	Este é um exemplo do uso do ambiente exe definido acima.
\end{exem}

Lorem ipsum dolor sit amet, consectetur adipiscing elit. Nunc dictum sed tortor nec viverra. consectetur adipiscing elit. Nunc dictum sed tortor nec viverra.

\begin{xdefinicao}
	Definimos o produto de ...
\end{xdefinicao}

Lorem ipsum dolor sit amet, consectetur adipiscing elit. Nunc dictum sed tortor nec viverra. consectetur adipiscing elit. Nunc dictum sed tortor nec viverra.

\section{Usando Questões}

Lorem ipsum dolor sit amet, consectetur adipiscing elit. Nunc dictum sed tortor nec viverra. consectetur adipiscing elit. Nunc dictum sed tortor nec viverra.

\begin{questao}
	\item Esta é a primeira questão com alguns itens:
		\begin{enumerate}
			\item Este é o primeiro item
			\item Segundo item
		\end{enumerate}
	\item Esta é a segunda questão:
		\begin{enumerate}
			\item Este é o primeiro item
			\item Segundo item
		\end{enumerate}
	\item Lorem ipsum dolor sit amet, consectetur adipiscing elit. Nunc dictum sed tortor nec viverra. consectetur adipiscing elit. Nunc dictum sed tortor nec viverra.
		\begin{enumerate}
			\item consectetur
			\item adipiscing
			\item Nunc
			\item dictum
		\end{enumerate}
\end{questao}

\section{Citações}

\subsection{Documentos com três autores}

Quando houver três autores na citação, apresentam se os três, separados por ponto e vírgula, caso estes estejam após o texto. Se os autores estiverem incluídos no texto, devem ser separados por vírgula e pela conjunção "e".

\citeautoronline{tresautores}

\cite{tresautores}

\subsection{Documentos com mais de três autores}
Havendo mais de três autores, indica-se o primeiro seguido da expressão \textit{et al.} (do latim \textit{et alli}, que significa e outros), do ano e da página.

\citeautoronline{quatroautores}

\cite{quatroautores}

\subsection{Documentos de vários autores}

Havendo    citações    indiretas de    diversos    documentos    de    vários    autores, mencionados  simultaneamente e  que  expressam  a  mesma  ideia,  separam-se  os  autores  por ponto e vírgula, em ordem alfabética.

\cite{tresautores, quatroautores}

\section{Notas de Rodap\'{e}}

Deve-se utilizar o sistema autor-data para as  citações no texto e o numérico para notas explicativas\footnote{Veja - se como exemplo desse tipo de abordagem o estudo de Netzer (1976)}. As notas de rodapé podem e devem ser alinhadas, a partir da segunda linha da mesma nota, abaixo da primeira letra da primeira palavra, de forma a destacar o expoente \footnote{Encontramos  esse  tipo  de  perspectiva  na  2ª  parte  do  verbete  referido  na  nota  anterior,  em  grande  parte  do estudo de Rahner (1962).} e sem espaço entre elas e com fonte menor (tamanho 10).
